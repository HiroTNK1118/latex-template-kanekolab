%! TEX root = ../main.tex
\documentclass[main]{subfiles}

\begin{document}
\chapter{評価実験}
本章では,本研究で提案した評価実験の実施方法およびその結果について述べる.

\section{実験目的}
本実験は,グロシについて,その有効性を検証することを目的とする.

\section{実験対象}
本実験は,グロシに所属するドヤコンガ5000兆名を対象として行った.

\section{実験方法}\label{how_to_jikken}
本実験は,実験1と実験2からなる.ヌッしかのこのこのここしたんたん.

\subsection{実験1}
実験1では,グロシって何ンガ?ヌッしかのこのこのここしたんたん.

\section{実験結果}
本節では,\ref{how_to_jikken}節で述べた実験1と実験2の実験結果を述べる.

\subsection{実験1の結果}
表\ref{table:jikken1_1}は,実験1において,グロシに成功した単語の個数である.

% 表の例
\begin{table}[htb]
    \centering
    \caption{できた個数}
    \label{table:jikken1_1}
    \begin{tabular}{crr} \hline
        科目 & 入力単語数 & 生成が成功した個数 \\ \hline
        グロシ & 334 & 334 \\
        SUSURU & 114 & 514 \\ \hline
    \end{tabular}
\end{table}

\end{document}