%! TEX root = ../main.tex
\documentclass[main]{subfiles}

\begin{document}
\chapter{緒言}
\section{背景}
学部生にとって,卒論を書くことは重要である\cite{Agarwal2012}.卒論の執筆には,\TeX の使用が一般的である.
\TeX のメリットとデメリットには,次のようなものがあげられる.

\begin{description}
    \item{メリット:} 
        \begin{itemize}
            \item レイアウトに惑わされず,内容の執筆に集中できること
            \item 美しい数式と組版が容易に手に入ること
        \end{itemize}
    \item{デメリット:}
        \begin{itemize}
            \item 環境構築と入力方法の学習に時間を要すること
        \end{itemize}
\end{description}

上記のように,\TeX にはメリットもデメリットもある.しかし,論文執筆やレポートなど,さまざまな場面で活用できる.また,上記のデメリットについても,インターネット上のたくさんの情報が,手助けとなるはずである\cite{Rao:AMCQG_Survey}.

文献は,bibファイルに,サイト等で取得したbibを追加していくことで利用できる.

こちらが,濃厚とんこつ豚無双さんの濃厚無双ラーメン,海苔トッピングです.うっひょ~~~~~!着席時,コップに水垢が付いていたのを見て,大きな声を出したら,店主さんからの誠意で,チャーシューをサービスしてもらいました.

俺の動画次第でこの店潰す事だってできるんだぞって事で,いただきま~~~~す!まずはスープから,コラ~!これでもかって位ドロドロの濃厚スープの中には,虫が入っており,怒りのあまり,卓上調味料を全部倒してしまいました~!

すっかり店側も立場を弁え,誠意のチャーシュー丼を貰った所で,お次に,圧倒的存在感の極太麺を,啜る~!殺すぞ~!

ワシワシとした食感の麺の中には,髪の毛が入っており,さすがのSUSURUも,厨房に入って行ってしまいました~!ちなみに,店主さんが土下座している様子は,ぜひサブチャンネルをご覧ください.

\section{目的}
%%4~5行
本研究では,グロシという概念について,様々な角度から検証を行う.旅人における圧倒的存在感のグロシを検証することが,本研究の目的である.

\section{本論文の構成}
以下,本論文は次のとおり構成される.まず,第2章では,関連研究や既存システムについて述べる.次に,第3章では,本研究で提案するシステムについて述べる.第4章では,システムに対する評価実験について述べる.第5章では,第4章で述べた実験の結果および考察について述べる.最後に,第6章で,本研究のまとめと今後の展望について述べる.
\end{document}