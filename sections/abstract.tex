%! TEX root = ../main.tex
\documentclass[main]{subfiles}

\begin{document}
{\centering 東京農工大学 工学部 知能情報システム工学科 \yearOfGraduation 年度 卒業論文 要旨\\
~\\
\titleJa
\\
\titleEn
\\
学籍番号 \studentNumber, 氏名 \myNameJa (\myNameEn)\\
提出日 \dateOfSubmissionAbs\\}
\vskip\baselineskip
本研究では、東京農工大学工学部知能情報システム工学科金子研究室における卒業論文および修士論文の執筆をターゲットとして、
効率的かつ簡便な執筆環境を提供することを目的とした {\LaTeX} テンプレートシステムを開発した。
本システムでは、実行環境となるDockerイメージと執筆環境であるDevContainerの設定、
さらに専用の {\LaTeX} クラスファイルおよび卒論用テンプレートを作成した。

本テンプレートは、DockerおよびDevContainerを用いた仮想環境上で実行する設計となっており、
{\TeX} Liveがインストール済みのDockerイメージを利用することで、セットアップを高速化するとともに必要なステップを最小限に抑えることができる。
加えて、Visual Studio CodeとDocker Desktopがインストールされているコンピュータがあれば、
専門的な知識がなくても簡単に環境構築が可能である。

{\LaTeX}環境(エンジン)は、Lua{\LaTeX} + B{\small{IB}}{\TeX} を基本としており、
理工系の日本語論文執筆に必要な \texttt{package} がすでに使用可能な状態で提供される。
また、各章ごとにファイルを分割する構成を採用しているため、大規模な文章の管理がしやすくなっている。
参考文献の引用方式には、IEEEスタイルをベースに日本語対応させた \texttt{jIEEEtran} を採用し、
さらに本研究室のスタイルに合うように小原さん(2023年度配属)が修正した \texttt{jIEEEtran.bst} ファイルを用いている。

本システムにより、金子研究室における卒業論文・修士論文の執筆作業がより効率化され、
{\LaTeX}初学者でもストレスなく質の高い文書作成が可能となることが期待される。
\end{document}