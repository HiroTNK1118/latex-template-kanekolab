\documentclass{classes/thesis}

\begin{document}

%%%%%%%% 著者名や題目はここで設定する %%%%%%%%%
\newcommand{\myNameJa}{
    %%%%%ここに日本語の著者名を書く
    %%%%%例:農工太郎
    農工太郎
}
\newcommand{\myNameEn}{
    %%%%%ここに英語の著者名を書く
    %%%%%例:Taro NOKO
    Taro NOKO
}
\newcommand{\studentNumber}{
    %%%%%学籍番号
    21266XXX
}
\newcommand{\yearOfEnrollment}{
    %%%%%入学年
    2021
}
\newcommand{\titleJa}{
    %%%%%日本語タイトル,1行で
    TUAT-EECS金子研究室の卒論・修論用LaTeX実行環境\&テンプレート
}
\newcommand{\titleJaLarge}{
    %%%%%日本語タイトルが長い場合,「\\[3mm]」で改行を含める(表紙でのみ使用)
    %%%%%通常は\titleJaと同じでOK
    TUAT-EECS金子研究室の卒論・修論用\\[3mm]
    LaTeX実行環境\&テンプレート
}
\newcommand{\titleEn}{
    %%%%%英語タイトル,1行で
    TUAT-EECS Kaneko Lab's LaTeX runtime environment \& templates for graduation thesis \& master's thesis
}
\newcommand{\titleEnLarge}{
    %%%%%英語タイトルが長い場合,「\\[3mm]」で改行を含める(表紙でのみ使用)
    %%%%%通常は\titleEnと同じでOK
    TUAT-EECS Kaneko Lab's LaTeX runtime environment \& templates\\[3mm]
    for graduation thesis \& master's thesis
}
\newcommand{\dateOfSubmission}{
    %%%%%卒論提出日
    2025年 2月 9日
}
\newcommand{\yearOfGraduation}{
    %%%%%卒業年度
    2024
}
\newcommand{\dateOfSubmissionAbs}{
    %%%%%要旨提出日
    2025年 2月 5日
}
\newcommand{\teacherName}{
    %%%%%指導教官名
    %%%%%例:農工 士郎 教授
    金子 敬一 教授
}

\frontmatter
\subfile{sections/title.tex}
\subfile{sections/abstract.tex}
\tableofcontents
\clearpage

\mainmatter
\subfile{sections/intro.tex}
\subfile{sections/relatedwork.tex}
\subfile{sections/method.tex}
\subfile{sections/experiment.tex}
\subfile{sections/discussion.tex}
\subfile{sections/conclusion.tex}

\backmatter
\subfile{sections/acknowledgment.tex}
\newpage
\subfile{sections/references.tex}
\newpage
\subfile{sections/appendix.tex}
\end{document}