\documentclass{classes/thesis}

%%%%%%%% 著者名や題目はここで設定する %%%%%%%%%
\newcommand{\myNameJa}{
    %%%%%ここに日本語の著者名を書く
    %%%%%例:農工太郎
    農工太郎
}
\newcommand{\myNameEn}{
    %%%%%ここに英語の著者名を書く
    %%%%%例:Taro NOKO
    Taro NOKO
}
\newcommand{\studentNumber}{
    %%%%%学籍番号
    21266XXX
}
\newcommand{\yearOfEnrollment}{
    %%%%%入学年
    2021
}
\newcommand{\titleJa}{
    %%%%%日本語タイトル,1行で
    金子研究室の卒論・修論用 {\LaTeX} 実行環境\&テンプレート
}
\newcommand{\titleJaLarge}{
    %%%%%日本語タイトルが長い場合,「\\[3mm]」で改行を含める(表紙でのみ使用)
    %%%%%通常は\titleJaと同じでOK
    金子研究室の卒論・修論用\\
    \vspace{3mm}
    {\LaTeX} 実行環境\&テンプレート
}
\newcommand{\titleEn}{
    %%%%%英語タイトル,1行で
    {\LaTeX} runtime environment \& template for Kaneko Lab's graduation \& master's thesis
}
\newcommand{\titleEnLarge}{
    %%%%%英語タイトルが長い場合,「\\[3mm]」で改行を含める(表紙でのみ使用)
    %%%%%通常は\titleEnと同じでOK
    {\LaTeX} runtime environment \& template\\
    \vspace{3mm}
    for Kaneko Lab's graduation \& master's thesis
}
\newcommand{\dateOfSubmission}{
    %%%%%卒論提出日
    2025年 2月 9日
}
\newcommand{\yearOfGraduation}{
    %%%%%卒業年度
    2024
}
\newcommand{\dateOfSubmissionAbs}{
    %%%%%要旨提出日
    2025年 2月 5日
}
\newcommand{\teacherName}{
    %%%%%指導教官名
    %%%%%例:農工 士郎 教授
    金子 敬一 教授
}

% マクロ\bstctlciteの定義:
% 文献リストで同一著者名が連続したとき-----のように省略されないようにする
% \begin{document}直下に宣言して使用すること
\makeatletter
\def\bstctlcite{\@ifnextchar[{\@bstctlcite}{\@bstctlcite[@auxout]}}
\def\@bstctlcite[#1]#2{\@bsphack
\@for\@citeb:=#2\do{%
\edef\@citeb{\expandafter\@firstofone\@citeb}%
\if@filesw\immediate\write\csname #1\endcsname{\string\citation{\@citeb}}\fi}%
\@esphack}
\makeatother

\begin{document}
\bstctlcite{IEEEexample:BSTcontrol}

% 前付け(ページ番号はローマ数字を用いる)
\pagenumbering{roman}
\subfile{sections/title.tex} % 表紙
\subfile{sections/abstract.tex} % 要旨
\tableofcontents % 目次
\cleardoublepage
\listoffigures % 図目次
\cleardoublepage
\listoftables % 表目次
\cleardoublepage

% 本文(ページ番号はアラビア数字を用いる)
\pagenumbering{arabic}
\subfile{sections/intro.tex} % 緒言
\subfile{sections/relatedwork.tex} % 関連研究
\subfile{sections/method.tex} % 提案手法
\subfile{sections/experiment.tex} % 評価実験
\subfile{sections/discussion.tex} % 考察
\subfile{sections/conclusion.tex} % 結言

% 後付け
\subfile{sections/acknowledgment.tex} % 謝辞
\newpage
\bibliography{references.bib} % 文献リスト
\bibliographystyle{jIEEEtran} % 文献リストのスタイルを指定
\newpage
\subfile{sections/appendix.tex} % 付録
\end{document}